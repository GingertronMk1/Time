\title{G54MRT Coursework 2 - The Ergonomotron}
\author{
  Jack Ellis \\
  psyje5@nottingham.ac.uk\\
  4262333\\\\
}
\date{}
\documentclass[12pt, a4paper]{report}
\usepackage{graphicx}
\graphicspath{{Images/}}

\usepackage{listings}
\lstset{
  basicstyle=\ttfamily
}

\begin{document}
\maketitle

\tableofcontents
\pagebreak

\section{Summary}
The broad idea of this project is to ensure a user is sitting properly and comfortably at their computer.
The project will use a combination of light sensors, distance sensors, and accelerometers to detect first whether or not the user is looking at their screen, and second whether or not they are craning their neck to do so in such a way as would lead to long-term damage.

\section{Background and Motivation}
Prolonged viewing of a computer screen can lead to many problems with regard to eye health, including dry eye syndrome\cite{dryeyes}, a condition in which the eyes do not produce enough tears, leading to irritation.
"Back and neck pain, headaches, and shoulder and arm pain are common computer-related injuries."\cite{vicgov}, and are related to poor posture when a user is at their workstation.
This project aims to alleviate these issues using a combination of light sensors and a 3-axis accelerometer.
The intended setting is, ultimately, in the workplace.
Many office jobs require employees to be sat at computers for large amounts of time, with no proper enforcement of regular breaks to ensure employee health.

\section{Related Work}
There is a lot of work currently centered around making computers more ergonomic and healthier to use, including standing desks, the adoption of blue light filters in software, and various monitor and laptop stands designed to raise the device screen such that the user is not craning their neck down risking damage.

\section{Design}

The system is designed around 4 sensors: 2 light sensors, and ultrasonic distance sensor, and a 3-axis gyroscope/accelerometer.
The light sensors will be arranged vertically at the front of the headgear, with a divider in between them to give an "above/beneath" light reading.
The distance sensor will similarly be mounted at the front of the headgear, likely next to the lower light sensor.
The gyroscope will be mounted wherever is convenient on the headgear.

\section{Implementation}

The prototype is based around a baseball cap, with the peak acting as the divider.
The sensors are attached using bent paperclips, and the Raspberry Pi is mounted to the top of the hat.
One light sensor is mounted above the peak, and the other hangs below it, while the gyroscope is mounted to the side of the hat.

\par

The code is split into two main sections

\section{Testing}

\section{Critical Reflection}

\begin{thebibliography}{0}
  \bibitem{dryeyes}
    \textit{Office screen work linked to dry eye syndrome}, 2014-06-18\\
    www.nhs.uk/news/lifestyle-and-exercise/office-screen-work-linked-to-dry-eye-syndrome\\
    Accessed 2018-03-08
  \bibitem{vicgov}
    \textit{Computer-related injuries}\\
    www.betterhealth.vic.gov.au/health/healthyliving/computer-related-injuries\\
    Accessed 2018-03-08
\end{thebibliography}

\end{document}
