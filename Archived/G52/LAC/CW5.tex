\title{G52LAC CW5}
\author{
  Jack Ellis\\
  psyje5@nottingham.ac.uk\\
  4262333
}
\documentclass[12pt]{article}
\usepackage{amssymb}
\usepackage{listings}
\lstset{
  frame=none,
  tabsize=2
}
\usepackage{array}
\usepackage[dvipsnames]{xcolor}
\begin{document}
\maketitle
\begin{itemize}
  \item \textbf{Question 1}
    \begin{itemize}
      \item \textbf{(a)}\\
        \begin{itemize}
          \item P1 is undecidable; the HP can be reduced to it.
          \item P2 is also undecidable, as all instances can be reduced to an instance of P1, which is undecidable.
          \item P4 is decidable; the normalisation problem for lambda-calculus is reducible to it.
        \end{itemize}
      \item \textbf{(b)}\\
        \begin{itemize}
          \item P1 is recursively enumerable; the Halting Problem is.\\
          \item P4 is recursively enumerable; it is stated as such in the initial information.\\
        \end{itemize}
      \item \textbf{(c)}\\
        \begin{itemize}
          \item P6 is not recursively enumerable; it is stated as such in the initial information.\\
          \item P7 is recursive; it is stated as such in the initial information.\\
          \item P8 is not; its complement is not recursively enumerable.\\
        \end{itemize}
    \end{itemize}

    \pagebreak
  \item \textbf{Question 2:}
    \begin{lstlisting}[language=haskell]
    evaluate :: SAT -> Assignment -> Bool
    evaluate (Var n) a = a !! n
    evaluate (Not s) a = not (evaluate s a)
    evaluate (And s1 s2) a = (evaluate s1 a) && (evaluate s2 a)
    evaluate (Or s1 s2) a = (evaluate s1 a) || (evaluate s2 a)
    \end{lstlisting}
    \verb|evaluate| uses Haskell's built-in Boolean functions to evaluate, and pattern-matches until it finds a \verb|Var|
    value.

    \pagebreak
  \item \textbf{Question 3}
    \begin{itemize}
      \item \textbf{(a)}
        \begin{lstlisting}[language=haskell]
varNum :: SAT -> Int
varNum s = maximum (varNum' s [])

varNum' :: SAT -> [Int] -> [Int]
varNum' (Var n) is = n:is
varNum' (Not s) is = varNum' s is
varNum' (And s1 s2) is = (varNum' s1 is) ++ (varNum' s2 is)
varNum' (Or s1 s2) is = (varNum' s1 is) ++ (varNum' s2 is)
        \end{lstlisting} 
        First we'll discuss \verb|varNum'|, which, similar to \verb|evaluate| uses pattern-matching to find only the
        \verb|Var| values in a formula, and generates a list of them. \verb|varNum| takes this list and finds the largest
        \verb|Var|, and returns it

      \item \textbf{(b)}
        \begin{lstlisting}[language=Haskell]
allAssign :: Int -> [Assignment]
allAssign n = sequence (replicate (n+1) [True, False])

allAssignSAT :: SAT -> [Assignment]
allAssignSAT s = allAssign (varNum s)
        \end{lstlisting}
        \verb|allAssign| creates a list of lists; namely a list of \verb|n [True,False| lists, where \verb|n| is passed in as
        an argument. \verb|sequence|, when applied to a list of lists in Haskell, returns the Cartesian Product, which is what
        we want. \verb|AllAssignSAT| was created to make the next few function easier.

      \item \textbf{(c)}
        \begin{center}
        \begin{lstlisting}[language=Haskell]
satisfiable :: SAT -> Bool
satisfiable s = or (map (evaluate s) (allAssignSAT s))
        \end{lstlisting}
        \end{center}
        \verb|satisfiable| takes a SAT, applies \verb|allAssign| to it, and then applies \verb|evaluate| to the resultant
        list. This returns a list of Bools, with the Bool saying whether or not that assignment solved the SAT. Applying
        \verb|or| to this list tells us if any of the assignments worked, and consequently if the SAT can be solved.

      \item \textbf{(d)}
        \begin{lstlisting}[language=Haskell]
solution :: SAT -> Maybe Assignment
solution s = do i <- elemIndex True ( map (evaluate s) allAss)
                return (allAss !! i)
             where allAss = allAssignSAT s
        \end{lstlisting}
        \verb|solution| uses \verb|elemIndex| to return a \verb|Maybe Int| with the index of the first \verb|True| in the list
        generated as part of \verb|satisfiable|. Using the \verb|do| notation, we can then use that \verb|Maybe Int| to index the list of
        all possible assignments returned by \verb|allAssign|, which will give us a working solution wrapped in a \verb|Maybe|.
    \end{itemize}

    \pagebreak
  \item \textbf{Question 4}
    \begin{itemize}
      \item \textbf{(a):}\\
        \verb|evaluate| looks at a SAT; if it's just a \verb|Var|, finds the value indexed by it, and returns that. On
        an input of size 1, it does 1 thing.\\
        If the SAT is a \verb|Not|, it applies itself to the SAT within, therefore it does one more step than it would
        for whatever the SAT inside would do.\\
        For \verb|And| or \verb|Or|, it \verb|evaluate|s the SATs within, then does the relevant Boolean operation on
        the result.\\
        The result of all this is that \verb|evaluate| runs in polynomial time; it does one thing for an input of size
        one, and more than two things for an input of size 2. Consequently, it is in NP, as it runs in polynomial time.
      \item \textbf{(b):}\\
        \verb|satisfiable| applies \verb|evaluate| (a polynomial algorithm), to a list, hence it repeatedly performs a
        polynomial-time algorithm, and consequently is exponential in running time.
      \item \textbf{(c):} SAT being NP-Complete means it is in NP, and every problem in NP can be reduced to it in
        polynomial time. 
    \end{itemize}
\end{itemize}
\end{document}
