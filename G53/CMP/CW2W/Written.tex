\title{G53CMP Coursework Part II}
\author{
  Jack Ellis\\
    psyje5@nottingham.ac.uk\\
    4262333
}
\date{}
\documentclass[12pt]{article}
\usepackage{graphicx}
\graphicspath{ {Images/} }
\usepackage{amsmath}
\usepackage{amssymb}
\usepackage{listings}
\lstset{
  basicstyle=\small\ttfamily,
    language=Haskell,
    tabsize=4,
    inputpath=../CW2/
}
\lstdefinestyle{Haskell}{
  \basicstyle=\small\tt
}
\newcommand{\lstin}[3]{
  \lstinputlisting[firstline=#1, lastline=#2]{#3}
}

\begin{document}
\maketitle
\tableofcontents
\pagebreak
\section{Task II.2}
\subsection{Character Literals}
\subsubsection{Type.hs}
This was a matter of adding the \verb|Character| option to the \verb|Type| data, and extending \verb|Eq| and \verb|Show|

\lstin{67}{81}{Type.hs}
\lstin{91}{97}{Type.hs}
\lstin{180}{185}{Type.hs}

\subsubsection{MTStdEnv.hs}
Here I added \verb|Character| to the \verb|TypeSym| array section of \verb|mtStdEnv|

\lstin{64}{68}{MTStdEnv.hs}

\subsection{Repeat-Until}

\subsubsection{MTIR.hs/PPMTIR.hs}
These were the same as the extensions to \verb|AST| and \verb|PPAST| from Part I

\lstin{84}{89}{MTIR.hs}
\lstin{69}{72}{PPMTIR.hs}

\subsubsection{TypeChecker.hs}
\verb|chkCmd| was extended to include the new \verb|CmdRepeat| MTIR representation

\lstin{120}{123}{TypeChecker.hs}

\pagebreak

\section{Task II.3}

\subsection{Print 1 To N}

\lstin{6}{19}{myTAMCode.hs}

First, an int is retrieved from the standard input. 
1 is then pushed onto the machine's stack. 
The label "comp" is applied for looping later. 
The contents of address Stack Base 0 are pushed onto the stack, as are then the contents of Stack Base 1. 
If 0's contents are smaller than 1's, "True" is pushed onto the stack, replacing both values. 
Otherwise, "False" is pushed, with the same effect. 
If the previous action returned "True", the program jumps to the end. 
Otherwise, Stack Base 1's contents are pushed to the stack, printed, and 1 added to them. 
The program jumps to the label "comp".
The label "end" is applied so the earlier jump has somewhere to go, and the program finishes.


\subsection{Factorial}

\lstin{21}{40}{myTAMCode.hs}

An int is retrieved from stdin. 
The subroutine "fac" is called. 
Its result is pushed to stdout. 
The program then halts.\par

The subroutine "fac" first pushes '1' to the stack. 
It then loads the value stored in Local Base -1, and compares them to 1. 
If they are smaller, it jumps to label "one" (this is the base case, ensuring any value less than one returns one).
Otherwise, the value in Local Base -1 is pushed to the top 2 positions in the stack, the top one then being subtracted by one. 
"fac" is called recursively here, what this serves to do is take the value and multiply it by all smaller natural numbers; this is the real factorial part. 
Once this recursive call finishes, MUL is called again, making the final factorial call, the "one" section is jumped, and the factorial value returned.

\subsection{getchr and putchr}

\subsubsection{LibMT.hs}

\lstin{169}{179}{LibMT.hs}

Adding these was largely a case of copying the parts for \verb|getint| and \verb|putint|, but modifying them with the GETCHR and PUTCHR MAT calls.

\subsubsection{MTStdEnv.hs}

And adding them here was again a matter of modifying the definitions for \verb|getint| and \verb|putint|, ensuring they were type-correct for Characters rather than Integers.

\lstin{90}{91}{MTStdEnv.hs}



\end{document}

