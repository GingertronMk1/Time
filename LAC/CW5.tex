\title{G52LAC CW5}
\author{
  Jack Ellis\\
  psyje5@nottingham.ac.uk\\
  4262333
}
\documentclass[12pt]{article}
\usepackage{amssymb}
\usepackage{listings}
\lstset{
  frame=none,
  tabsize=2
}
\usepackage{array}
\usepackage[dvipsnames]{xcolor}
\begin{document}
\maketitle
\begin{itemize}
  \item \textbf{Question 1}
    \begin{itemize}
      \item \textbf{(a)}\\
        P1 is undecidable; the HP can be reduced to it.
        P2 is also undecidable, as all instances can be reduced to an instance of P1, which is undecidable.\\
        P4 is decidable; the normalisation problem for \lambda-calculus is reducible to it.
      \item \textbf{(b)}\\
        P1 is recursively enumerable; the Halting Problem is.\\
        P4 is recursively enumerable; it is stated as such in the initial information.\\
      \item \textbf{(c)}\\
        P6 is not recursively enumerable; it is stated as such in the initial information.\\
        P7 is recursive; it is stated as such in the initial information.\\
        P8 is not; its complement is not recursively enumerable.\\
    \end{itemize}

    \pagebreak
  \item \textbf{Question 2:}
    \lstinputlisting[language=haskell, firstline=30, lastline=34]{CW5.hs}
    \verb|evaluate| uses Haskell's built-in Boolean functions to evaluate, and pattern-matches until it finds a \verb|Var|
    value.

    \pagebreak
  \item \textbf{Question 3}
    \begin{itemize}
      \item \textbf{(a)}
        \lstinputlisting[language=haskell, firstline=36, lastline=43]{CW5.hs}
        First we'll discuss \verb|varNum'|, which, similar to \verb|evaluate| uses pattern-matching to find only the
        \verb|Var| values in a formula, and generates a list of them. \verb|varNum| takes this list and finds the largest
        \verb|Var|, and returns it

      \item \textbf{(b)}
        \lstinputlisting[language=haskell, firstline=45, lastline=49]{CW5.hs}
        \verb|allAssign| creates a list of lists; namely a list of \verb|n [True,False| lists, where \verb|n| is passed in as
        an argument. \verb|sequence|, when applied to a list of lists in Haskell, returns the Cartesian Product, which is what
        we want.

      \item \textbf{(c)}
        \lstinputlisting[language=haskell, firstline=51, lastline=52]{CW5.hs}
        \verb|satisfiable| takes a SAT, applies \verb|allAssign| to it, and then applies \verb|evaluate| to the resultant
        list. This returns a list of Bools, with the Bool saying whether or not that assignment solved the SAT. Applying
        \verb|or| to this list tells us if any of the assignments worked, and consequently if the SAT can be solved.

      \item \textbf{(d)}
        \lstinputlisting[language=haskell, firstline=54, lastline=57]{CW5.hs}
        \verb|solution| uses \verb|elemIndex| to return a \verb|Maybe Int| with the index of the first \verb|True| in the list
        generated as part of \verb|satisfiable|. Using the \verb|do| notation, we can then use that \verb|Maybe Int| to index the list of
        all possible assignments returned by \verb|allAssign|, which will give us a working solution wrapped in a \verb|Maybe|.
    \end{itemize}

    \pagebreak
  \item \textbf{Question 4}
    \begin{itemize}
      \item \textbf{(a):}\\
      \item \textbf{(b):}
      \item \textbf{(c):} SAT being NP-Complete means it is in NP, and every problem in NP can be reduced to it in
        polynomial time.
    \end{itemize}
\end{itemize}
\end{document}
